%%%	 Use this file as a template for your submission to your conference, and to prepare
%%%	the final version of your paper, if accepted, for the conference Proceedings.
%%%	You can use it "as is" for your initial submission, but alterations may be required if
%%%	there is a preliminary Proceedings for distribution at the meeting, and additional changes
%%%	are mandatory for the final version for publication in the official conference Proceedings.
%%%. 	If you are preparing your paper in final form for the conference proceedings,
%%%. 	add the option "final" to \documentclass{entics} below

\documentclass[twoside,11pt]{entics}
\usepackage{enticsmacro}
\usepackage{graphicx}
\usepackage[all]{xy}
%\usepackage{url}
\sloppy
% The following is enclosed to allow easy detection of differences in
% ascii coding.
% Upper-case    A B C D E F G H I J K L M N O P Q R S T U V W X Y Z
% Lower-case    a b c d e f g h i j k l m n o p q r s t u v w x y z
% Digits        0 1 2 3 4 5 6 7 8 9
% Exclamation   !           Double quote "          Hash (number) #
% Dollar        $           Percent      %          Ampersand     &
% Acute accent  '           Left paren   (          Right paren   )
% Asterisk      *           Plus         +          Comma         ,
% Minus         -           Point        .          Solidus       /
% Colon         :           Semicolon    ;          Less than     <
% Equals        =3D           Greater than >          Question mark ?
% At            @           Left bracket [          Backslash     \
% Right bracket ]           Circumflex   ^          Underscore    _
% Grave accent  `           Left brace   {          Vertical bar  |
% Right brace   }           Tilde        ~

% A couple of exemplary definitions:

\newcommand{\Nat}{{\mathbb N}}
\newcommand{\Real}{{\mathbb R}}

%%%Fill in the following%%%%
%%%%%%%%	Your Corresponding Editor will provide the following information:
\def\conf{MFPS 2024} 	%%Fill in the acronym for your conference (with year)
\volume{NN}			%Fill in the ENTICS volume number here
\def\volu{NN}			% and here
\def\papno{nn}			%Fill in your paper number here
%%%%%%%%Please fill in the following information:
\def\lastname{List Authors' Lastnames Here} %Lastnames appear in the running header
%									on odd pages. If more than three authors, use et al
\def\runtitle{List a short title here} %Short title appears in the running header on even pages.
%%
\def\copynames{A. Author1, B. Author2, \ldots}    %% Fill in the first initial and last name of the authors
%%%%%%%%%%%%%%%%%%%		claiming Creative Commons copyright here.
\def\CCB{CC@symbol}		%%%%%%	Also, be sure the correct Creative Commons
%%%%%%%%%%%%%					copyright symbol is chosen - see Section 2 of
%%%%%%%%%%		https://mirrors.concertpass.com/tex-archive/fonts/ccicons/ccicons.pdf
\begin{document}
%%%Note the beginning and end of the frontmatter section that starts here%%%%%
\begin{frontmatter}
  \title{An Example Paper\thanksref{ALL}} 						%%Title here and the
  \thanks[ALL]{General thanks to everyone who should be thanked.}   %%Text of \thanks[ALL} here..
  %%%%%%%%%%%%%%%%%%%%%%%%%%%%			This Thanks is optional.
  %%%%Now the author(s) names(s)%%%%%
  \author{My Name\thanksref{a}\thanksref{myemail}}	%%Note NO SPACE between
  \author{My Co-author\thanksref{b}\thanksref{coemail}}		%last name and \thanksref{...}
  %%%Next come the addresses%%%%
  \address[a]{My Department\\ My University\\				%or between \thanksrefs...
    My City, My Country}
  \thanks[myemail]{Email: \href{mailto:myuserid@mydept.myinst.myedu} {\texttt{\normalshape
        myuserid@mydept.myinst.myedu}}}
  %%%Note: if both authors share same institution, only list the address once, after the second
  %%%author.
  %%%There also is a link from the first author to the co-author's address to show how to list
  %%%affiliations to more than one institution, when needed.
  \address[b]{My Co-author's Department\\My Co-author's University\\
    My Co-author's City, My Co-author's Country}
  \thanks[coemail]{Email:  \href{mailto:couserid@codept.coinst.coedu} {\texttt{\normalshape
        couserid@codept.coinst.coedu}}}
  \begin{abstract}
    This is a short example to demonstrate the \LaTeX\ style files used to prepare
    submissions for \conf, and for preparing the final version of papers
    accepted for the conference.
    These Proceedings will appear in the \emph{Electronic Notes in Theoretical Informatics and
      Computer Science} (ENTICS), a new series that is published under the
    auspices of Epsiciences.org, the French national organization that supports open
    access overlay publications in a broad range of scientific disciplines.
  \end{abstract}
  \begin{keyword}
    Please list keywords from your paper here, separated by commas.
  \end{keyword}
\end{frontmatter}
\section{Introduction}\label{intro}
This short example serves as a template for using the \LaTeX\ macro package for
use in preparing submissions for \conf. Authors of papers accepted for
presentation at \conf\ will be asked to provide updated versions for the
\emph{Preliminary Proceedings} of \conf, which will be distributed to the
participants at the meeting.

The \emph{Proceedings} of \conf\ will be published after the meeting. They will
appear later this year, as a volume in the forthcoming series \textit{\jour}.
ENTICS is an open access, online series published using the support and
facilities of Episciences.org~\url{https://episciences.org}. You may be
familiar with Episciences.org, which also hosts \emph{Logical Methods in
  Computer Science}. ENTICS is the first series devoted to online publication of
conference proceedings to be hosted by Episciences.org. The webpages for ENTICS
will be on the Episciences platform, with information about the series, details
about each volume, together with links to the papers in each volume. ENTICS is
an overlay series, with papers published in the series located on one of the
repositories, the \href{https://arxiv.org}{CORR archive} hosted at Cornell
University, the \href{https://hal.archives-ouvertes.fr}{HAL archive} hosted in
the French repository, or the \href{https://zenodo.org}{Zenado} archive, hosted
at \href{https://home.cern}{CERN}. Detailed instructions, together with the
updated style files for preparing the final version of papers accepted for
\conf, will be provided later.

The ENTICS \LaTeX\ macros are derived from the macros used by its predecessor,
\emph{Electronic Notes in Theoretical Computer Science}, which was the first
solely online series devoted to conference proceedings. That series ceased
publication at the end of 2020.

This package consists of three files:
\begin{description}
  \item[\texttt{entics.bst},] a style file for formatting bibliographic entries,
  \item[\texttt{entics.cls},] the basic style file, and
  \item[\texttt{enticsmacro.sty},] a macro file containing the
        definitions of some of the theorem-like environments and a few other
        tidbits.
\end{description}
The formatting these style files impose should \emph{not} be altered
-- they are used to establish and maintain a uniform format for the
papers accepted for presentation at \conf. In particular,
the font choices should not be changed -- ENTICS uses the standard CMR Type 1 fonts
for the text body of papers appearing in the series, and while other fonts may be
used to specialized purposes, such as fonts to support special math symbols,
changing the text font is not allowed. In particular, the \texttt{Times Roman} fonts should \emph{not} be used for the text body, and the \texttt{txfonts} should be avoided because they break some of the macros used in this package.

Additional macro files can be added using \verb+\usepackage{...}+. The file
\texttt{entics\-macro.sty} \emph{must} be included in the list, as is done at
the start of the source file for this document. Nonstandard macros can be used,
\emph{as long as they do not interfere with} \texttt{entics.cls} \emph{or}
\texttt{enticsmacro.sty}. It's a good idea to list \texttt{enticsmacro.sty} as
the last \verb+\usepackage{...}+ called in preliminary portion of your
document, since \LaTeX\ will report errors if there are conflicts with other
packages you want to use.

The ENTICS package requires \LaTeX 2e, which is needed to produce pdf files as
the final output. Since ENTICS is published online, we also utilize the
\texttt{hyperref} package, which supports active hyperlinks in pdf documents.
While the use of pdf\LaTeX\ is preferred for producing a pdf file as the final
output, authors who need to include Postscript graphics in their papers may
wish to utilize the older combination of \LaTeX 2e and dvips, In order to
support this, entics.cls includes the \texttt{ifpdf} package to differentiate
between pdf\LaTeX\ on one hand, or \LaTeX 2e followed by dvips and then ps2pdf,
to produce the final output file.

\section{Frontmatter}
The biggest difference between a \LaTeX\ style file such as
\texttt{article.sty} and the file \texttt{entics.cls} is that
\texttt{entics.cls} requires the title, author's name or names, abstract,
keywords and ``thanks'' all to be included within the \texttt{frontmatter}
environment. You'll notice this at the beginning of the source file for this
template. Also, you'll also notice that the usual \verb+\maketitle+ is absent;
it no longer is needed. The ENTICS style package automatically generates the
title, author's name and address, and related material at the beginning of the
paper.

One important point to note is that \LaTeX\ is unable to generate addresses or
footnotes from within the \verb+\author{...}+ field in the \texttt{frontmatter}
environment, so the macro requires the use of \verb+\thanksref{xx}+ within the
\verb+\author{...}+ field, followed by a corresponding \verb+\thanks[xx]{...}+
as a separate entry outside the \verb+\author{...}+. field. Thus,
\verb+\thanksref{xx}+ places a superscript $^{\texttt{xx}}$ next to the
author's last name, which provides an active link to the footnote created by
\verb+\thanks[xx]{..}+. Note that both the \verb+\thanksref{...}+ and the
\verb+\thanks[xx]{...}+ must appear within the \texttt{frontmatter}
environment.

The ENTICS macro package provides two alternatives for listing authors'
addresses. The simplest method is to list each author and his or her address in
turn, as is done in this example. But, if there are several authors and two or
more share the same address (but not all authors are at this address), then the
preferred method is to list all authors first, each with a separate
\verb+\author{...}+, and then to list the addresses. Then each author can be
linked to the appropriate address(es) by using a \verb+\thanksref{xx}+ to
indicate that author has an affiliation to the address listed as
\verb+\address[xx]{...}+. This is illustrated in this example, where the first
author has a link to the co-author's \verb+\address{...}+ to indicate there is
an affiliation at that address as well. This mechanism is handy when there are
several authors and some of them share the same affiliation.
%The preferred method is to list all the authors' names first, each with a separate \verb+\author{...}+, and then to provide a list of addresses, again with each having a separate \verb+\address{...}+ listing, and to use the \verb+\thankrsref{xx}+ and \verb+\thanks[xx]{...}+ mechanism to link each author to the appropriate address(es).

Also, notice that acknowledgment of support (as the contents of a
\verb+\thanks{...}+) can be done by a separate listing of, e.g.,
\verb+\thanks[NSF]{To the NSF}+ with the optional argument -- \verb+[NSF]+ --
used for a \verb+\thanksref{NSF}+ that is attached to those authors
acknowledging that support.

It is important that the \verb+\thanks+ not be included within the scope of
\verb+\author{}+ or of \verb+\title{}+, but it must be within the scope of the
environment \texttt{frontmatter}.

The \texttt{frontmatter} section also contains an abstract environment, as well
as one for key words. These will be required for the final version of an
accepted paper.

More details about added terms such as \verb+\collab+ can be found in
\texttt{inst.dvi}, if they are needed.

Also, notice that the command \verb+\lastname{Please list Your Lastname Here}+
that appears \emph{before} the \verb+\begin{document}+. This command should
contain the last names of the authors of the paper. If there are no more than
three authors, then they should be listed with the word ``and'' between the
last two; if more than three authors collaborated on the paper, then the first
author only should be listed, together with \verb+\emph{et al}+. This command
creates the headline for each page after page 1.

\section{Sectioning and Environments}
The ENTICS package supports the standard sectioning commands,
\verb+\section,\subsection, \paragraph+ and \verb+\subparagraph.+ The numbering
scheme used is one under which Theorem 1.2.3 is the third numbered item in
second subsection of the first section of the paper. In order to facilitate
cross-references, all of the named environments given below are numbered, and
all use the same number scheme. For example, there are:
\begin{definition}
  A file is \emph{derived} from another if it is obtained with some modifications from the original file.
\end{definition}

\begin{theorem}
  The file \texttt{\normalshape entics.cls} is derived from
  \texttt{\normalshape entcs.cls}.
\end{theorem}
\begin{proof}
  This is clear from the similarity of the output to the output from
  the ENTCS style files.
\end{proof}

Starting a proof with a descriptive word, such as ``sketch'', is supported by
using the \verb+\begin{proof*}...\end{proof*}+ environment, as in

\begin{proof*}{Proof (Sketch)}
  This can be derived from simple observations.
\end{proof*}

Note that unnumbered sectioning also is supported, as in \verb+\section*{...}+,
etc. But these sections will not support cross references within the document,
so they should be used only for material that doesn't need to be
cross-referenced.

Again, the purpose of \texttt{entics.cls} and \texttt{enticsmacro.sty} is to
impose a uniform format on the output file, so please don't include spurious
spacing -- such as \verb+\vskip{...}+ or such -- within the document. It's also
inappropriate to end a line with \verb+\\+, which starts the following line at
the left margin. Doing this creates a very haphazard,, ugly format with ragged
right justification. The goal is to have an easily recognizable format
associated with the series \emph{Electronic Notes in Theoretical Informatics
  and Computer Science}.

%All of the usual features of \LaTeX\ are available with these
%style files -- it is only the formatting that has been rigorously
%defined.

The file \texttt{enticsmacro.sty} contains additional information that is
needed to typeset a paper. It also has the definitions of the $\cal AMS$
\texttt{euler} and \texttt{blackboard bold} fonts builtin. If you want to use
symbols for the natural numbers, the reals, etc., then we prefer that you use
the blackboard bold fonts, and not plain bold fonts. This is accomplished by
using the \verb+\mathbb+ font, as in $\Nat$ or $\Real$.

The names of theorem-like environments are provided in
\texttt{enticsmacro.sty}. With the exception of the Algorithm environment, the
names of all of these are full name, rather than a shortened version. The
environments provided and their names are
\begin{itemize}
  \item \verb+\begin{theorem} ... \end{theorem}+ for Theorems,
  \item \verb+\begin{lemma} ... \end{lemma}+ for Lemmas,
  \item \verb+\begin{corollary} ... \end{corollary}+ for Corollaries,
  \item \verb+\begin{proposition} ... \end{proposition}+ for
        Propositions,
  \item \verb+\begin{criterion} ... \end{criterion}+ for Criteria,
  \item \verb+\begin{alg} ... \end{alg}+ for Algorithms,
  \item \verb+\begin{definition} ... \end{definition}+ for Definitions,
  \item \verb+\begin{conjecture} ... \end{conjecture}+ for Conjectures,
  \item \verb+\begin{example} ... \end{example}+ for Examples,
  \item \verb+\begin{problem} ... \end{problem}+ for Problems,
  \item \verb+\begin{remark} ... \end{remark}+ for Remarks,
  \item \verb+\begin{note} ... \end{note}+ for Notes,
  \item \verb+\begin{claim} ... \end{claim}+ for Claims,
  \item \verb+\begin{summary} ... \end{summary}+ for Summary,
  \item \verb+\begin{case} ... \end{case}+ for Cases, and
  \item \verb+\begin{ack} ... \end{ack}+ for Acknowledgements.
\end{itemize}

For example,

\begin{algorithm}[h]
  \begin{alg} The algorithm for preparing a submission.\\
    Step 1:  Write the paper\\
    Step 2: Format it with the ENTICS macro package\\
    Step 3:  Submit the paper to \conf. \\
  \end{alg}
\end{algorithm}

In addition, the \verb+itemize, enumerate+ and \verb+description+ environments
also are supported, as usual.

\section{Tables, Figures and Graphics}

Tables and figures are supported, and can be cross-referenced using
\verb+\label+s. For example, here is Table~\ref{mytable}:

\begin{table}[htp]
  \caption{This is My Table}
  \begin{center}
    \begin{tabular}{|c|c|c|}
      \hline X & Y & Z \\
      \hline
      A        & B & C \\
      \hline
    \end{tabular}\label{mytable}
  \end{center}
\end{table}

Figures have a similar environment. We use Figure~\ref{myfigure} below to
illustrate the figure environment with a graphic image; note we have included
\verb+\usepackage{graphicx}+ at the beginning of this file. The graphicx
package supports \verb+.png, .jpg,+ and \verb+.pdf+ images, which can be
inserted in the document as follows:

\begin{figure}[h!]
  \begin{center}
    \includegraphics[scale=.2]{chess.pdf}
    \caption{A Chess Board}\label{myfigure}
  \end{center}
\end{figure}
Notice we have used the \verb+scale=.xin+ option to scale graphics so that the image is completely within the space for the text body, and doesn't ``bleed over" into the margins. We also have used it to assure the image fits close to the desired place in the document. Also, note the use of \verb+\begin{center} ... \end{center}+ in both tables and figures. This centers the body of the table or figure. Without it, the body will be left-justified.

Both \verb+tables+ and \verb+\figures+ have a \verb+\caption{...}+, followed by
a \verb+\label{...}+ for cross-referencing (if the \verb+\label{...}+ is not
after to the \verb+\caption{...}+, then the reference numbering will be wrong
-- try moving \verb+\label{...}+ before \verb+\caption{...}+ in either the
table or figure above to see). Traditionally, tables have the caption at the
top, while figures have them at the bottom, but either way works.

In addition to inserting graphics into a document, \verb+entics.sty+ also
supports diagram drawing packages, such as X\raisebox{-2pt}{Y}-pic,
\texttt{tikzit} and Paul Taylor's diagrams package. For example, here's a
diagram from the X\raisebox{-2pt}{Y}-pic Users Guide -- note the inclusion of
\verb+\usepackage[all]{xy}+ at the beginning of this file:

$$\xymatrix{
  U \ar@/_/[ddr]_y \ar@/^/[drr]^x
  \ar@{.>}[dr]|-{(x,y)}			\\
  & X \times_Z Y \ar[d]^q \ar[r]_p
  & X \ar[d]_f \\
  & Y \ar[r]^g & Z }$$

\subsection{Particulars about {\normalshape \texttt{.pdf} files}} We now require that \texttt{.pdf} files be provided for publication
online. A \texttt{.pdf} file is viewable by Adobe's Acrobat$^{\tiny
      \copyright}$ viewer, which can be configured to load automatically within a
browser. Viewing a properly formatted \texttt{.pdf} file with Acrobat$^{\tiny
      \copyright}$ allows the cross-references and links to URLs to be active. In
fact, Elsevier utilizes \texttt{.pdf} files in order to take better advantage
of the web's capabilities.

But one point we want to emphasize is that you should be sure to use Type 1
fonts when you typeset your \LaTeX\ source file. These fonts are scalable,
meaning that they carry information that allows the devise viewing the final
output to scale the fonts to suit the viewer being used -- from an onscreen
viewer such as Adobe's Acrobat$^{\tiny \copyright}$ Reader, to printing the
file on a printer. You can tell if you have used the right fonts by viewing the
final output on your machine. It the fonts look grainy, then you have not used
Type 1 fonts. They can be located at the CTAN archive \url{http://www.ctan.org}
-- they are public domain fonts, and don't cost anything to add them to your
system.

%Assuming you have Type 1 fonts available, then there are there methods
%for producing \texttt{.pdf} files.
%
%\paragraph{Using \texttt{dvips} and \texttt{ps2pdf}}
%We list this option first since it appears to be the most reliable and
%the easiest to use, especially if you include embedded PostScript
%graphics (\texttt{.eps} files) in your source file. Simply run \LaTeX
%2e on your source file, then apply \texttt{dvips} to produce a
%PostScript file, and finally apply \texttt{ps2pdf} to obtain a
%\texttt{.pdf} file.
%
%\paragraph{The \texttt{DVIPDFM} utility}
%Another easy method for producing acceptable \texttt{.pdf} files is
%via the utility \texttt{dvipdfm}. This utility is included in
%distributions of Mik\TeX, which runs on Windows machines, but it
%probably needs to be added to your te\TeX\ distribution, if you are
%running \LaTeX\ on a UNIX machine. The utility and precise information
%about installing it on your system can be found at the web page
%\href{http://gaspra.kettering.edu/dvipdfm/}{\tt
%  http://gaspra.kettering.edu/dvipdfm/}. In essence, this utility
%converts a \texttt{.dvi} file into a \texttt{.pdf} file. So, one can
%first prepare the \texttt{.dvi} file using \LaTeX, and then apply the
%utility \texttt{dvipdfm} to produce the needed \texttt{.pdf}
%file.\footnote{ \emph{Beware}! The utility \texttt{dvipdf} does
%  \emph{not} produce acceptable \texttt{.pdf} files, and should not be
%  used. Only \texttt{dvipdfm} should be used to produce \texttt{.pdf}
%  files.} This utility makes inclusion of graphics particularly simple
%-- those that are included in the \LaTeX\ source file are simply
%converted to the \texttt{.pdf} format. As we note below, things are
%not so simple with the second alternative, which is to use pdf\LaTeX.

\paragraph{pdf\LaTeX}
Since the required output for files published in ENTICS is a \texttt{.pdf}
file, the easiest way to produce the output is by using pdf\LaTeX\ to process
the source file. pdf\LaTeX\ is now included in \LaTeX\ distributions available
from the standard CTAN sites \url{http://www.ctan.org}.

There is one aspect of pdf\LaTeX\ that can create a problem. It's sometimes
useful to produce a graphic image using PostScript, but pdf\LaTeX\ cannot
process such files. The way to include them is to create the desired graphic
image as a \verb+.eps+ file\footnote{eps stands for \emph{embedded
    PostScript}.}, and then to use the utility \verb+epstopdf+ to convert the
output to a \verb+.pdf+ file. This approach generates high quality graphics
output. The resulting \verb+.pdf+ file then can be included by using the
\verb+graphicx+ package, for example. The chess board example above was
produced in this way. Here as well is a color image using pdf\LaTeX\ and
\verb+.pdf+ input file:\\
\begin{center}
  \includegraphics[scale=.25]{tigre.pdf}
\end{center}

\section{References and Cross-references}
All the cross-referencing facilities of \LaTeX\ are supported, so one can use
\verb+\ref{}+ and \verb+\cite{}+ for cross-references within the paper and for
references to bibliographic items. While the \textbf{References} can be
composed with \verb+\begin{thebibliography}...\end{thebibliography}+, it is
better to use Bib\TeX\ to compile the bibliography from a \verb+.bib+ file.
Further details about the formatting requirements are given in
Section~\ref{references} below.

The package \texttt{hyperref} is automatically loaded by \texttt{entics.cls},
and this makes all the cross-references within the document ``active'' when the
pdf file of the paper is viewed with a pdf viewer supporting hyperlinks, such
as Adobe's Acrobat$^{\tiny \copyright}$ Reader. The format for including a link
is simple: simply insert \verb+\url{URL}{text}+ where \texttt{URL} is the URL
to which you want the link to point, and \texttt{text} is the text you want to
be highlighted, which when clicked upon will bring up the desired web page. You
also can use \verb+\url{link}+ where \texttt{link} is the URL that is being
pointed to.

\section{Summary}  The ENTICS macro package is relatively easy to use and provides a uniform
layout for all the papers that appear in ENTICS.

\begin{problem}
Finish your paper and submit it to EasyChair on time!
\end{problem}

%%When you have finished preparing your paper, send a copy of the
%%\emph{source file}, together with any macro files that are needed to
%%your Program Chairman.  If the files are extensive, you can place
%%copies in the \texttt{pub/incoming} sub-directory of the ftp directory
%%on the machine indicated by your Program Chairman using anonymous ftp.
%%If you do this, please send me email to alert me that the file(s) are
%%here.

%%\paragraph{Assigning Volume / Issue Numbers}
%%One additional point worth mentioning is that ENTICS is moving to
%%ScienceDirect, Elsevier's main platform for publishing electronic
%%series, Because ScienceDirect must publish entire volumes at the same
%%time, we have changed the procedure for preparing final versions so
%%that volume numbers will not be assigned until the final versions are
%%ready. Guest Editors will now have to receive the final version of all
%%papers in their \emph{Proceedings} before a volume and issue number
%%will be assigned for the \emph{Proceedings}. Even with the move to
%%ScienceDirect, the reference scheme already used for publications in
%%ENTICS -- \texttt{http://www.elsevier/nl/locate/entcs/}
%%\texttt{NNnn.html} remains the valid way to cite papers published in
%%ENTICS, where \texttt{NN} denotes the number of the volume, and
%%\texttt{nn} denotes the issue number.  Publications consisting of an
%%entire volume should use \texttt{1} as the issue number.
%%
%%\paragraph{Copyright Transfer Forms}
%%One result of the move to ScienceDirect is that the corresponding
%%author of each paper published in ENTICS must complete an online Copyright
%%Transfer Form. Information about this is sent to the corresponding author shortly
%%before the paper is published online.
%%
%%\paragraph{Publication of Final Versions}
%%Because ScienceDirect cannot easily accommodate changes to published
%%material, the Proceedings in its entirety must be ready before it can
%%be published. This is one reason why the volume and issue number is
%%not assigned until the final versions of all papers have been sent to
%%the Guest Editors for final processing.

\section{Bibliographical references}\label{references}
ENTICS employs the \texttt{entcs.bst} style file for bibliographic references.
In this format, references are listed in alphabetical order, according the the
first author's last name, and are sequentially numbered. In addition, the first
item in each reference is the first author's last name, followed by their
initials or first name, and then the other authors names are listed, etc, It's
easiest to use Bib\TeX\ to generate the bibliography from a \verb+.bib+ file to
implement these and the other requirements. There is a file
\texttt{example.bib} containing a number of citations, with illustrations for
how to enter articles (\cite{Civin-Yood,Freyd,Easdown-Munn}), books
(\cite{Roscoe,Clifford-Preston,Weyl} and PhD theses (\cite{Shehadah}), as well
as many others.

The rules for references are the following:
\begin{itemize}
  \item Authors' names should be listed in alphabetical order, with the first author's
        last name being the first listing, followed by the author's initials or first
        name, and with the other authors names listed as \emph{first name, last name}.
  \item Titles of articles in journals should be in \emph{emphasized} type.
  \item Titles of books, monographs, etc.\ should be in quotations.
  \item Journal names should be in plain roman type.
  \item Journal volume numbers should be in boldface type, with the year of publication
        immediately following in roman type, and enclosed in parentheses.
  \item A DOI (digital object identifier) should be provided for each references that
        has one. Links to URLs for those that don't have a DOI should be ``active'' and
        the URL itself should be in typewriter font. The DOI / URL should be the last
        item in the reference.
  \item Article listings should include page numbers.
\end{itemize}
The criteria are illustrated in the following.

\bibliographystyle{./entics}
\bibliography{example}
%\begin{thebibliography}{10}\label{bibliography}
%\bibitem{cy} Civin, P., and B. Yood, \emph{Involutions on Banach
%    algebras}, Pacific J. Math. \textbf{9} (1959), 415--436, \url{https://msp.org/pjm/1959/9-2/p07.xhtml}
%
%\bibitem{cp} Clifford, A. H., and G. B. Preston, ``The Algebraic
%  Theory of Semigroups,'' Math. Surveys \textbf{7}, Amer. Math. Soc.,
%  Providence, R.I., 1961. Electronic ISBN: 978-1-4704-1234-0 \url{https://www.ams.org/books/surv/007.1/surv007.1-endmatter.pdf}
%
%\bibitem{f} Freyd, Peter, Peter O'Hearn, John Power, Robert Tennent
%  and Makoto Takeyama, \emph{Bireflectivity}, Electronic Notes in
%  Theoretical Computer Science {\bf 1} (1995), DOI: \url{https://doi.org/10.1016/S1571-0661(04)00026-X}
%
%\bibitem{em2} Easdown, D., and W. D. Munn, \emph{Trace functions on
%    inverse semigroup algebras}, Bulletin of the Australian Mathematical Society \textbf{52}(03):359 - 372
%DOI: \url{http://dx.doi.org/10.1017/S0004972700014854}
%
%\bibitem{r} Roscoe, A. W., ``The Theory and Practice of Concurrency,''
%  Prentice Hall Series in Computer Science, Prentice Hall Publishers,
%  London, New York (1198), 565pp. ISBN 0-13-674409-5 With associated web site\\
%  \url{http://www.comlab.ox.ac.uk/oucl/publications/books/concurrency/}
%%  \href{http://www.comlab.ox.ac.uk/oucl/publications/books/concurrency/}
%%  {\texttt{http://www.comlab.ox.ac.uk/oucl/publications/books/concurrency/}}.
%
%\bibitem{s} Shehadah, A. A., ``Embedding theorems for semigroups with
%  involution, `` Ph.D.  thesis, Purdue University, Indiana, 1982. \url{https://docs.lib.purdue.edu/dissertations/AAI8300958/}
%
%\bibitem{w} Weyl, H., ``The Classical Groups,'' 2nd Ed., Princeton U.
%  Press, Princeton, N.J., 1946. ISBN: 9780691057569
%
%\end{thebibliography}

\appendix
\section{An appendix}
Any appendices should be included after the references, as is done here. The
appendix starts with the command \verb+\appendix+, after which sections can be
included; these are lettered, rather than numbered. \section*{Another appendix}
One can also use \verb+\section*{...}+ to create an appendix without a letter
attached.
\end{document}
