\clearpage
\section{Appendix}

\allowdisplaybreaks

\paragraph{Implementation}

The implementation of \superfluid can be found at \url{https://github.com/kontheocharis/superfluid}.

\paragraph{Formalisation} \label{app:formalisation}

The Agda formalisation of the developments of this paper at \url{https://github.com/kontheocharis/rep-agda}.

\subsection{Definition of \lambdamltt}

For reference, we define Martin-Löf type theory with $\univ :
\univ$, $\Pi$, $\Sigma$, propositional equality and unit. We omit the
definition of the substitution calculus and equality coercions
(see \cite[5.1.2]{Castellan2019-qo}). We use de-Brujin indices, keeping weakening
implicit, and abuse notation for substitutions of terms: $A[t]$ for $A[\mta{id},
t]$.

\newcommand{\refl}{\mta{refl}}
\newcommand{\fst}{\mta{fst}}
\newcommand{\snd}{\mta{snd}}
\newcommand{\pair}{\mta{pair}}
\newcommand{\app}{\mta{app}}
\newcommand{\lam}{\mta{lam}}
\newcommand{\J}{\mta{J}}


\begin{figure}[h]
\begin{mathpar}
% Universe formation
\inferrule[Univ-Form]
  { }
  {\Gamma \vdash \univ \type}

% Element formation
\inferrule[El-Form]
  {\Gamma \vdash a : \univ}
  {\Gamma \vdash \El a \type}

% Element formation
\inferrule[Univ-Intro]
  {\Gamma \vdash A \type}
  {\Gamma \vdash \code A : \univ}

% Pi type formation
\inferrule[Pi-Form]
  {\Gamma \vdash A \type \\ \Gamma, x : A \vdash B \type}
  {\Gamma \vdash (x : A) \to B \type}

% Lambda abstraction (introduction)
\inferrule[Pi-Intro]
  {\Gamma, x : A \vdash b : B}
  {\Gamma \vdash \lambda\ x.\ b : (x : A) \to B}

% Function application (elimination)
\inferrule[Pi-Elim]
  {\Gamma \vdash f : (x : A) \to B \\ \Gamma \vdash a : A}
  {\Gamma \vdash f\ a : B[a]}

% Equality formation
\inferrule[Eq-Form]
  {\Gamma \vdash A \type \\ \Gamma \vdash a : A \\ \Gamma \vdash b : A}
  {\Gamma \vdash a \equiv_A b \type}

% Reflexivity (introduction)
\inferrule[Eq-Intro]
  {\Gamma \vdash a : A}
  {\Gamma \vdash \refl\ a : a \equiv_A a}

% J (equality elimination)
\inferrule[Eq-Elim]
  {\Gamma \vdash A \type \\
   \Gamma, a : A, b : A, a \equiv_A b \vdash P \type \\
   \Gamma, a : A \vdash r : P[a, a, \refl\ a] \\
   \Gamma \vdash a : A \\ \Gamma \vdash b : A \\ \Gamma \vdash p : a \equiv_A b}
  {\Gamma \vdash \J\ P\ d\ p : P[a, b, p]}

% Unit type formation
\inferrule[Unit-Form]
  { }
  {\Gamma \vdash \top \type}

% Unit introduction
\inferrule[Unit-Intro]
  { }
  {\Gamma \vdash \mta{tt} : \top}

% Sigma type formation
\inferrule[Sigma-Form]
  {\Gamma \vdash A \type \\ \Gamma, x : A \vdash B \type}
  {\Gamma \vdash (x : A) \times B \type}

% Pair introduction
\inferrule[Sigma-Intro]
  {\Gamma \vdash a : A \\ \Gamma \vdash b : B[a]}
  {\Gamma \vdash (a, b) : (x : A) \times B}

% First projection (elimination)
\inferrule[Sigma-Elim-Fst]
  {\Gamma \vdash p : (x : A) \times B}
  {\Gamma \vdash \fst\ p : A}

% Second projection (elimination)
\inferrule[Sigma-Elim-Snd]
  {\Gamma \vdash p : (x : A) \times B}
  {\Gamma \vdash \snd\ p : B[\fst\ p]}
\end{mathpar}
\caption{Typing rules for \lambdamltt.}
\end{figure}

\begin{figure}[h]
\begin{minipage}[t]{0.5\textwidth}
\begin{align*}
  &\boxed{\text{Universes}} \\
  & {\El (\code A) = A} \\
  & {\code (\El t) = t} \\[1em]
  &\boxed{\text{$\Pi$ types}} \\
  & { (\lambda\ x.\ b)\ a = b[a]} \\
  & { \lambda\ x.\ (f\ x) = f }
\end{align*}
\end{minipage} \qquad
\begin{minipage}[t]{0.5\textwidth}
\begin{align*}
&\boxed{\text{Equality and unit}} \\
& { \J\ P\ r\ (\refl\ a) = r\ a } \\
& { x = \mta{tt}} \\[1em]
&\boxed{\text{$\Sigma$ types}} \\
& { \fst\ (a, b) = a} \\
& { \snd\ (a, b) = b} \\
& { (\fst\ p, \snd\ p) = p }
\end{align*}
\end{minipage}
\caption{Definitional equality rules for \lambdamltt, omitting substitution rules
such as $(\El a) [\sigma] = \El (a[\sigma])$.}
\end{figure}

\subsection{Definition of \lambdadata}

The language \lambdadata is the extension of \lambdamltt. by the rules in
\cref{fig:data-rules,fig:repr-rules}. Below we present some additional
definitional equality rules of $\MRepr$ that make it commute with most of the
syntax, as well as some propositional equalities that are justified by the translation $\R$.

\begin{figure}[H]
  \begin{minipage}[t]{0.4\textwidth}%
  \begin{align*}
  & \boxed{\text{$\eta$ rules}} \\
  & \Munrepr\ (\Mrepr\ t) = t \\
  & \Mrepr\ (\Munrepr\ t) = t \\[1em]
  & \boxed{\text{Stability under substitution}} \\
  & \Mrepr\ {(t[\sigma])} = (\Mrepr\ {t})[\sigma] \\
  & \Munrepr\ {(t[\sigma])} = (\Munrepr\ {t})[\sigma] \\
  & \MRepr\ {(T[\sigma])} = (\MRepr\ {T})[\sigma] \\[1em]
  & \boxed{\text{Compatibility with $\Pi$ types}} \\
  & \MRepr\ {((x : A) \to B)} = (x : A) \to \MRepr\ B \\
  & \Mrepr\ {(\lambda\ x.\ b)} = \lambda\ x.\ (\Mrepr\ b) \\
  & \Munrepr\ {(\lambda\ x.\ b)} = \lambda\ x.\ (\Munrepr\ b) \\
  & \Mrepr\ (f\ a) = (\Mrepr\ f)\ a \\
  & \Munrepr\ (f\ a) = (\Munrepr\ f)\ a
  \end{align*}
  \end{minipage}\qquad
  \begin{minipage}[t]{0.5\textwidth}%
  \begin{align*}
  & \boxed{\text{Compatibility with universes}} \\
  & \MRepr\ {\univ} = \univ \\
  & \Mrepr\ {(\Code A)} = \Code A \\
  & \Munrepr\ {(\Code A)} = \Code A \\[1em]
  & \boxed{\text{Compatibility with equality}} \\
  & \MRepr\ (a \equiv_A b) = \Mrepr\ a \equiv_{\MRepr A} \Mrepr\ b \\
  & \Mrepr\ (\mta{refl}\ a) = \mta{refl}\ (\Mrepr\ a) \\
  & \Munrepr\ (\mta{refl}\ a)  = \mta{refl}\ (\Munrepr\ a) \\
  & \Mrepr\ (\mta{J}\ P\ d\ p) = \mta{J}\ (\mta{Repr}\ P)\ (\mta{repr}\ d)\ p \\
  & \Munrepr\ (\mta{J}\ (\mta{Repr}\ P)\ d\ p) = \mta{J}\ P\ (\mta{unrepr}\ d)\ p \\[1em]
  & \boxed{\text{Compatibility with eliminators}} \\
  & \Mrepr\ (\mta{elim}_S\ M\ \beta\ \delta\ x) \\ &\qquad = \mta{elim}_S\ (\MRepr\ M)\ (\Mrepr\ \beta)\ \delta\ x \\
  & \Munrepr\ (\mta{elim}_S\ (\MRepr\ M)\ \beta\ \delta\ x)\\ &\qquad = \mta{elim}_S\ M\ (\Munrepr\ \beta)\ \delta\ x
  \end{align*}
  \end{minipage}%
  \caption{(\mech{https://github.com/kontheocharis/rep-agda/blob/e6bd34adaab630f5787c63a95fa86869f6c19da4/TT/Repr.agda\#L41})\
  Definitional compatibility rules for $\MRepr$. Similar rules are given for
  $\Sigma$ and $\top$ in the formalisation. In this version, $\MRepr$ only applies to codomains
  of functions which aligns with the substitution rule. However, it is also
  possible to formulate it as \\ $\MRepr\ ((x : A) \to B) = (x : \MRepr\ A) \to \MRepr\ B[\Munrepr\ x]$.}
  \label{fig:lambdaind-repr-coherence-pi-univ}
\end{figure}

\begin{figure}[H]
\vspace{2em}
  \begin{align*}
  & \mta{repr-ctor}_{S.O}\ \{\gamma\}\ \nu : \\ & \qquad \Mrepr\ (\Mctor_{S.O}\ \{\gamma\} \nu) \equiv \gamma.\alpha_O\ (\Mrepr\ \nu) \\[1em]
  & \mta{elim-equiv}_{S}\ \{\gamma\}\ M\ \beta\ \delta\ x :  \\ & \qquad
  \mta{elim}_{S}\ \{\gamma\}\ M\ \beta\ \delta\ x \equiv \gamma.\kappa\ (\delta.\ x.\ M[\delta, \Munrepr\ x])\ (\Mrepr^*\ \beta)\ \delta\ (\Mrepr\ x)
  \end{align*}
  \caption{Additional propositional equalities for $\MRepr$ on constructors and eliminators. Here, $\Mrepr^*$ applies $\MRepr$ on all the recursive
    arguments of a displayed algebra. The rule $\mta{elim-equiv}_{S}\ \{\gamma\}\ M\ \beta\ \delta\ x$ is also derivable internally by case analysis
    on $x$.}
\end{figure}
