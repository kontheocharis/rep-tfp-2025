\section{Appendix}

The metatheory is presented internally to an extensional dependent type theory
with an internal universe $\Set$, dependent function types $(a : A) \to B$ and 
dependent pair types $(a : A) \times B$. The metatheory also supports 
quotient inductive-inductive definitions, which are used to define the syntaxes
of the languages presented in this paper.

\subsection{The language $\lambdaind$}

\begin{figure}[H]
  \begin{minipage}[t]{0.5\textwidth}
  \begin{align*}
  \Ty &: \Con \to \Set \\
  \_[\_] &: \Ty\ \Delta \to \Sub\ \Gamma\ \Delta \to \Ty\ \Gamma\\
  \univ &: \Ty\ \Gamma \\
  \Pi &: (A : \Ty\ \Gamma) \to \Ty\ (\Gamma \rhd A) \to \Ty\ \Gamma \\
  \El{} &: \Tm\ \Gamma\ \univ \to \Ty\ \Gamma 
  \end{align*}%
  \end{minipage}
  \begin{minipage}[t]{0.5\textwidth}%
  \begin{align*}
  \Tm &: (\Gamma : \Con) \to \Ty\ \Gamma \to \Set \\
  \_[\_] &: \Tm\ \Delta\ A \to (\sigma : \Sub\ \Gamma\ \Delta) \to \Tm\ \Gamma\ A[\sigma] \\
  \pi_2 &: (\sigma : \Sub\ \Gamma\ (\Delta \rhd A)) \to \Tm\ \Gamma\ A[\pi_1 \sigma] \\
  \lambda &: \Tm\ (\Gamma \rhd A)\ B \to \Tm\ \Gamma\ (\Pi\ A\ B) \\
  \ap{} &: \Tm\ \Gamma\ (\Pi\ A\ B) \to \Tm\ (\Gamma \rhd A)\ B \\
  \Code{} &: \Ty\ \Gamma \to \Tm\ \Gamma\ \univ 
  \end{align*}
  \end{minipage}
  \caption{The type and term formers of the base layer of $\lambdaind$.}
  \label{fig:lambdaind-base-formers}
\end{figure}

\begin{figure}[H]
  TODO
  \caption{Equations in the base type system.}
\end{figure}

\begin{figure}[H]
  \begin{minipage}[t]{0.5\textwidth}
  \begin{align*}
  \Sig &: \Set \\
  \epsilon &: \Sig \\
  \rhd &: (\Sigma : \Sig) \to (A : \Item\ \Sigma) \to \Sig \\[1em]
  \Item &: \Sig \to \Set \\
  \mta{ind} &: \Ind\ \Sigma \to \Item\ \Sigma \\
  \mta{def} &: (A : \Ty\ (\Sigma, \epsilon)) \to \Tm\ (\Sigma, \epsilon)\ A \to \Item\ \Sigma \\
  \mta{post} &: \Ty\ (\Sigma, \epsilon) \to \Item\ \Sigma \\
  \end{align*}
  \end{minipage}
  \begin{minipage}[t]{0.5\textwidth}
  \begin{align*}
  \Loc &: \Sig \to \Set \\
  \epsilon &: \Loc\ \Sigma \\
  \rhd &: (\Delta : \Loc\ \Sigma) \to \Ty\ (\Sigma,\Delta) \to \Loc\ \Sigma \\[1em]
  \Con &: \Set \\
  \Con &:= (\Sigma : \Sig) \times \Loc\ \Sigma
  \end{align*}
  \end{minipage}
  \caption{Signatures, items, local contexts and contexts in $\lambdaind$.}
  \label{fig:lambdaind-signatures-contexts}
\end{figure}

\begin{figure}[H]
  \begin{minipage}[t]{\textwidth}
  \begin{align*}
  \Sub &: \Con \to \Con \to \Set \\
  \mta{id} &: \Sub\ \Gamma\ \Gamma \\
  \epsilon &: \Sub\ (\Sigma, \Delta)\ (\Sigma, \epsilon) \\
  \rhd &: (\sigma : \Sub\ (\Sigma,\Delta_1)\ (\Sigma,\Delta_2)) \to \Tm\ (\Sigma, \Delta_1)\ A[\sigma] \to \Sub\ (\Sigma,\Delta_1)\ (\Sigma,\Delta_2 \rhd A) \\
  \circ &: \Sub\ \Gamma_1\ \Gamma_2 \to \Sub\ \Gamma_2\ \Gamma_3 \to \Sub\ \Gamma_1\ \Gamma_3 \\
  \pi_1 &: \Sub\ (\Sigma,\Delta_1)\ (\Sigma, \Delta_2 \rhd A) \to \Sub\ (\Sigma,\Delta_1)\ (\Sigma,\Delta_2) 
  \end{align*}
  \end{minipage}
  \caption{Substitutions between contexts in $\lambdaind$.}
  \label{fig:lambdaind-substitutions}
\end{figure}

\begin{figure}[H]
  \begin{alignat*}{2}
  & \Ind &&: (\Sigma : \Sig) \to \Set \\
  & \Ind\ \Sigma &&:= (P : \Loc\ \Sigma) \times (\Xi : \Tel\ (\Sigma, P)) \times \Theory\ (\Sigma, P)\ \Xi \\[1em]
  & \Theory &&: (\Gamma : \Con) \to (\Xi : \Tel\ \Gamma) \to \Set \\
  & \Pi_{\mta{ext}} &&: (A : \Set) \to (A \to \Theory\ \Gamma\ \Xi) \to \Theory\ \Gamma\ \Xi \\
  & \Pi_{\mta{int}} &&: (A : \Ty\ \Gamma) \to \Theory\ (\Gamma \rhd A)\ \Xi^+ \to \Theory\ \Gamma\ \Xi \\
  & \Pi_{\mta{rec}} &&: \Tms\ \Gamma\ \Xi \to \Theory\ \Gamma\ \Xi \to \Theory\ \Gamma\ \Xi \\
  & \mta{ret} &&: \Tms\ \Gamma\ \Xi \to \Theory\ \Gamma\ \Xi
  \end{alignat*}
  \caption{Inductive types, theories and operations in $\lambdaind$.}
  \label{fig:lambdaind-inductive-theories}
\end{figure}

\begin{figure}[H]
  \begin{alignat*}{2}
  &\mta{In} &&: \Theory\ \Gamma\ \Xi \to \Ty\ (\Gamma \unrhd \Xi) \to \Tms\ \Gamma\ \Xi \to \Set \\
  &\mta{In}\ (\Pi_{\mta{ext}}\ A\ B)\ X\ \xi &&:= (x : X) \times \mta{In}\ (B\ x)\ X\ \xi \\
  &\mta{In}\ (\Pi_{\mta{int}}\ A\ B)\ X\ \xi &&:= (a : \Tm\ \Gamma\ A) \times \mta{In}\ B [\langle a \rangle]\ X\ \xi \\
  &\mta{In}\ (\Pi_{\mta{rec}}\ \xi'\ B)\ X\ \xi &&:= \Tm\ \Gamma\ (X [\langle \xi' \rangle]) \times \mta{In}\ B\ X\ \xi \\
  &\mta{In}\ (\mta{ret}\ \xi')\ X\ \xi &&:= \xi = \xi'
  \end{alignat*}
  \caption{Interpretation of operation inputs (i.e. constructor arguments) and outputs (constructor return indices).}
  \label{fig:lambdaind-operation-interp}
\end{figure}

\begin{figure}[H]
  \begin{alignat*}{2}
  &\mta{Alg} &&: \Theory\ \Gamma\ \Xi \to \Ty\ (\Gamma \unrhd \Xi) \to \Set \\
  &\mta{Alg}\ (\Pi_{\mta{ext}}\ A\ B)\ X &&:= (x : X) \to \mta{Alg}\ (B\ x)\ X \\
  &\mta{Alg}\ (\Pi_{\mta{int}}\ A\ B)\ X &&:= (a : \Tm\ \Gamma\ A) \to \mta{Alg}\ B [\langle a \rangle]\ X \\
  &\mta{Alg}\ (\Pi_{\mta{rec}}\ \xi'\ B)\ X &&:= \Tm\ \Gamma\ (X [\langle \xi' \rangle]) \to \mta{Alg}\ B\ X \\
  &\mta{Alg}\ (\mta{ret}\ \xi)\ X &&:= X\ \xi \\[1em]
  \end{alignat*}
  \caption{Interpretation of operation inputs (i.e. constructor arguments) and outputs (constructor return indices).}
\end{figure}

\newcommand{\Msub}[1]{\langle{#1}\rangle}

\begin{figure}[H]
  \begin{align*}
  &\mta{Motive} : \Theory\ \Gamma\ \Xi \to \Ty\ (\Gamma \unrhd \Xi) \to \Set \\
  &\mta{Motive}\ T\ X := \Ty\ (\Gamma \unrhd \Xi \rhd X) \\[1em]
  &\mta{Method} : \{T : \Theory\ \Gamma\ \Xi\} \to \mta{Motive}\ T\ X \to \mta{Alg}\ T\ X \to \Set \\
  &\begin{alignedat}{2}
  &\mta{Method}\ \{\Pi_{\mta{ext}}\ A\ B\}\ X\ M\ \alpha &&:= (x : A) \to \mta{Method}\ (B\ x)\ X\ M\ (\alpha\ x) \\
  &\mta{Method}\ \{\Pi_{\mta{int}}\ A\ B\}\ X\ M\ \alpha &&:= (a : \Tm\ \Gamma\ A) \to \mta{Method}\ B [\langle a \rangle]\ X\ M\ (\alpha\ a) \\
  &\mta{Method}\ \{\Pi_{\mta{rec}}\ \xi\ B\}\ X\ M\ \alpha &&:= (x : \Tm\ \Gamma\ X [\langle \xi \rangle]) \to \Tm\ \Gamma\ M [\langle \xi ; x \rangle] \to \mta{Method}\ B\ X\ M\ (\alpha\ x) \\
  &\mta{Method}\ \{\mta{ret}\ \xi\}\ X\ M\ \alpha &&:= \Tm\ \Gamma\ M [\langle \xi ; \alpha \rangle] 
  \end{alignedat} \\[1em]
  &\mta{Section} : \{T : \Theory\ \Gamma\ \Xi\} \to \mta{Motive}\ T\ X \to \Set \\
  &\mta{Section}\ \{T\}\ \{X\} := (\xi : \Tms\ \Gamma\ \Xi) \to (x : \Tm\ \Gamma\ (X[\Msub \xi])) \to \Tm\ \Gamma\ M[\Msub {\xi; x}] \\[1em]
  \end{align*}
  \caption{Methods}
  \label{fig:lambdaind-motive-methods}
\end{figure}

\newcommand{\fmap}{\;\langle\$\rangle\;}

\begin{figure}[H]
  \begin{alignat*}{2}
  & \mta{data} &&: (\mta{ind}\ (P, \Xi, T) \in \Gamma) \to (p : \Tms\ \Gamma\ P) \to (\xi : \Tms\ \Gamma\ \Xi[\langle p \rangle]) \to \Ty\ \Gamma \\ 
  & \mta{ctor} &&: (i : \mta{ind}\ (P, \Xi, T) \in \Gamma) \to (p : \Tms\ \Gamma\ P) \\
  &            && \to (a : \mta{In}\ T\ (\mta{data}\ i\ p)\ \xi) \to \Tm\ \Gamma\ (\mta{data}\ i\ p\ xi) \\[0.5em]
  & \mta{elim} &&: (i : \mta{ind}\ (P, \Xi, T) \in \Gamma) \to (p : \Tms\ \Gamma\ P) \\ 
  &            && \to (M : \mta{Motive}\ T\ (\mta{data}\ i\ p)) \to (m : \mta{Methods}\ M) \to \mta{Section}\ M \\
  & \mta{match} &&: (\mta{elim}\ i\ p\ M\ m)[\langle \xi, \mta{ctor}\ i\ p\ n\ a \rangle] = (m\,!\,n)[\langle a , \mta{elim}\ i\ p\ M\ m \fmap a \rangle]
  \end{alignat*}
  \caption{Data type, constructor and eliminator terms in $\lambdaind$.}
  \label{fig:lambdaind-data-ctor}
\end{figure}

\subsection{The language $\lambdarep$}

\begin{figure}[H]
  \begin{minipage}[t]{\textwidth}%
  \begin{alignat*}{2}
  & \MRepr &&: \Ty\ \Gamma \to \Ty\ \Gamma \\
  & \Mrepr &&: \Tm\ \Gamma\ T \to \Tm\ \Gamma\ (\MRepr\ T) \\
  & \Munrepr &&: \Tm\ \Gamma\ (\MRepr\ T) \to \Tm\ \Gamma\ T
  \end{alignat*}
  \end{minipage}\\
  \begin{minipage}[t]{0.5\textwidth}%
  \begin{alignat*}{2}
  \end{alignat*}
  \end{minipage}%
  \begin{minipage}[t]{0.5\textwidth}%
  \begin{alignat*}{2}
  \end{alignat*}
  \end{minipage}%
  \caption{New term and type formers in $\lambdarep$.}
  \label{fig:lambdarep-new-formers}
\end{figure}

\begin{figure}[H]
  \begin{minipage}[t]{0.5\textwidth}%
  \begin{alignat*}{2}
  & \mta{reprr} &&: \Munrepr\ (\Mrepr\ t) \equiv t \\
  & \mta{reprl} &&: \Mrepr\ (\Munrepr\ t) \equiv t \\[1em]
  & \MRepr\text{-}\Pi &&: \MRepr\ {(\Pi\ T\ U)} \equiv \Pi\ T\ (\MRepr\ U) \\ 
  & \Mrepr\text{-}\lambda &&: \Mrepr\ {(\lambda\ u)} \equiv \lambda\ (\Mrepr\ u) \\
  & \Munrepr\text{-}\lambda &&: \Munrepr\ {(\lambda\ u)} \equiv \lambda\ (\Munrepr\ u) \\
  & \Mrepr\text{-}@ &&: \Mrepr\ (\ap f) \equiv \ap {(\Mrepr\ f)} \\
  & \Munrepr\text{-}@ &&: \Munrepr\ (\ap f) \equiv \ap {(\Munrepr\ f)}
  \end{alignat*}
  \end{minipage}%
  \begin{minipage}[t]{0.5\textwidth}%
  \begin{alignat*}{2}
  & \MRepr\text{-}\univ &&: \MRepr\ {\univ} \equiv \univ \\
  & \Mrepr\text{-}\Code{} &&: \Mrepr\ {(\Code T)} \equiv \Code T \\
  & \Munrepr\text{-}\Code{} &&: \Munrepr\ {(\Code T)} \equiv \Code T \\[1em]
  & \MRepr[] &&: \MRepr\ {(T[\sigma])} \equiv (\MRepr\ {T})[\sigma] \\
  & \Mrepr[] &&: \Mrepr\ {(t[\sigma])} \equiv (\Mrepr\ {t})[\sigma] \\
  & \Munrepr[] &&: \Munrepr\ {(t[\sigma])} \equiv (\Munrepr\ {t})[\sigma] 
  \end{alignat*}
  \end{minipage}%
  \caption{Coherence of representation terms with substitutions, $\Pi$-types and
  universes. The terms $\MRepr\ (\El{t})$, $\Mrepr\ (\pi_2 \sigma)$ and
  $\Munrepr\ (\pi_2 \sigma)$ do not reduce, and there is no rule to collapse
  repeated invocations of $\MRepr{}$, $\Mrepr{}$ and $\Munrepr{}$.}
  \label{fig:lambdaind-repr-coherence-pi-univ}
\end{figure}

\begin{figure}[H]
  \begin{alignat*}{2}
  & \Rep &&: (\Sigma : \Sig) \to \Item\ \Sigma \to \Set \\
  & \mta{ind-rep} &&: \mta{IndRep}\ F \to \Rep\ \Sigma\ (\mta{ind}\ F) \\
  & \mta{def-rep} &&: (x : \Tm\ (\Sigma, \epsilon)\ A) \to \Tm\ (\Sigma, \epsilon)\ (x = t) \to \Rep\ \Sigma\ (\mta{def}\ A\ t) \\
  & \mta{post-rep} &&: \Tm\ (\Sigma, \epsilon)\ A \to \Rep\ \Sigma\ (\mta{post}\ A) \\[1em]
  \end{alignat*}
  \caption{Implementations of items and representations of signatures in $\lambdarep$.}
  \label{fig:lambdaind-impls-reprs}
\end{figure}

\subsection{The translation}

\begin{figure}[H]
  \begin{alignat*}{2}
  & \mta{Impl} &&: \Sig_\rep \to \Set \\
  & \mta{Impl}\ \Sigma &&:= (I \in \Sigma) \to \mta{Maybe}\ (\Rep\ I)
  \end{alignat*}
\end{figure}

\begin{figure}[H]
  \begin{alignat*}{2}
  & \R &&: (\Sigma : \Sig_\rep) \to \Impl\ \Sigma \to \Sig_\ind \\
  & \R\ (\epsilon)\ f &&:= \epsilon \\
  & \R\ (\Sigma \rhd I)\ f &&:= \kwd{if}\ f\ (\Sigma \rhd I)\ \mta{here} = \mta{just}\ r\ \kwd{then}\ \Sigma\ \kwd{else}\ (\Sigma \rhd I) \\
  \end{alignat*}
\end{figure}

\begin{figure}[H]
  \begin{alignat*}{2}
  & \R_{f : \Impl\ \Sigma} &&: \Ty_\rep\ (\Sigma, \Delta)\ \to \Ty_\ind\ (\R_f\ (\Sigma, \Delta)) \\
  & \R_f\ (T[\sigma]) &&:= (\R_f\ T)[\R_f\ \sigma] \\
  & \R_f\ (\univ) &&:= \univ \\
  & \R_f\ (\Pi\ A\ B) &&:= \Pi\ (\R_f\ A)\ (\R_f\ B) \\
  & \R_f\ (\El t) &&:= \El{(\R_f\ t)} \\
  & \R_f\ (\mta{data}\ i\ p) &&:= \kwd{if}\ f\ \Sigma\ i = \mta{just}\ (\mta{ind-rep}\ r)\ \kwd{then}\ \emb{r}\ \kwd{else}\ (\mta{data}\ i\ p) \\
  & \R_f\ (\MRepr\ T) &&:= \R_f\ T \\[1em]
  & \R_{f : \Impl\ \Sigma} &&: \Tm_\rep\ (\Sigma, \Delta)\ T \to \Tm_\ind\ (\R_f\ (\Sigma, \Delta))\ (\R_f\ T) \\
  & \R_f\ (t[\sigma]) &&:= (\R_f\ t)[\R_f\ \sigma] \\
  & \R_f\ (\pi_2\ \sigma) &&:= \pi_2\ (\R_f\ \sigma) \\
  & \R_f\ (\lambda\ t) &&:= \lambda\ (\R_f\ t) \\
  & \R_f\ (t\ @) &&:= (\R_f\ t)\ @ \\
  & \R_f\ (\code T) &&:= \code (\R_f\ T) \\
  & \R_f\ (\mta{ctor}\ i\ p\ n\ a) &&:= \kwd{if}\ f\ \Sigma\ i = \mta{just}\ (\mta{ind-rep}\ r)\ \kwd{then}\ \emb{r}\ \kwd{else}\ (\mta{ctor}\ i\ p\ n\ a) \\
  & \R_f\ (\mta{elim}\ i\ p\ M\ m) &&:= \kwd{if}\ f\ \Sigma\ i = \mta{just}\ (\mta{ind-rep}\ r)\ \kwd{then}\ \emb{r}\ \kwd{else}\ (\mta{elim}\ i\ p\ M\ m) \\
  & \R_f\ (\Mrepr\ t) &&:= \R_f\ t \\
  & \R_f\ (\Munrepr\ t) &&:= \R_f\ t
  \end{alignat*}
\end{figure}
  % & \R\ (\epsilon) &&:= \epsilon \\
  % & \R\ (\Sigma \rhd I) &&:= \R \Sigma \unrhd \R I \\[1em]
  % & \R &&: \Item_\rep\ \Sigma \to \mta{Items}_\ind\ (\R \Sigma) \\
  % & \R\ (\mta{ind}\ I) \mid \mta{ind}\ I &&:= \epsilon \\
  % & \R\ (\Sigma \rhd I) &&:= \R \Sigma \unrhd \R I

% \begin{figure}[h]
%   \begin{mathpar}
%   \inferrule[Repr-Ctor-Id]
%   {
%     \Sreprvar{\ctorlab c\ \Pi}{\kappa} \in \Sigma
%   }
%   {\Sigma \mid \Gamma \vdash \repr{(\ctorlab{c}\ \delta\ \pi)} \equiv \kappa [\delta, \pi] : A[\delta, \xi[\pi]]} \and
%   \inferrule[Repr-Data-Id]
%   {
%     \Sreprvar{\datalab D\ \Delta\ \Xi}{A} \in \Sigma
%   }
%   {\Sigma \mid \Gamma \vdash \istype{\Repr{(\datalab{D}\ \delta\ \psi)} \equiv A [\delta,\psi]}}
%   \end{mathpar}
%   \caption{Definitional equalities for $\Repr{}$ and $\repr{}$ relating to data
%   types and constructors with defined representations. Similar equalities hold
%   for representations of global function definitions and eliminators, albeit
%   propositionally.}
% \end{figure}

  % \begin{minipage}[t]{\textwidth}
  % \begin{alignat*}{2}
  % &\llbracket\_\rrbracket &&: \Op\ \Gamma\ \Xi \to \Ty\ (\Gamma \unrhd \Xi) \to \Ty\ (\Gamma \unrhd \Xi) \\
  % &\llbracket \Pi_{\mta{ext}}\ A\ B \rrbracket\ X &&:= \Pi\ A\ (\llbracket B \rrbracket X) \\
  % &\llbracket \Pi_{\mta{int}}\ \xi\ B \rrbracket\ X &&:= \Pi\ (X [\langle \xi \rangle])\ (\llbracket B \rrbracket X)^+ \\
  % &\llbracket \mta{ret}\ \xi \rrbracket\ X &&:= X [\langle \xi \rangle]
  % \end{alignat*}
  % \end{minipage}

  % & \Rep &&: \Sig \to \Sig \to \Set \\
  % & \mta{id} &&: \Rep\ \Sigma\ \Sigma \\
  % & \epsilon &&: \Rep\ \Sigma\ \epsilon \\
  % & \rhd &&: (\rho : \Rep\ \Sigma_1\ \Sigma_2) \to \Rep\ \Sigma_1\ I[\rho] \to \Rep\ \Sigma_1\ (\Sigma_2\rhd I) \\
  % & \circ &&: \Rep\ \Sigma_1\ \Sigma_2 \to \Rep\ \Sigma_2\ \Sigma_3 \to \Rep\ \Sigma_1\ \Sigma_3 \\
  % & \pi_1 &&: \Rep\ \Sigma_1\ (\Sigma_2 \rhd I) \to \Rep\ \Sigma_1\ \Sigma_2