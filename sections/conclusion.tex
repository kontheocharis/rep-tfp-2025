\section{Conclusions and Future Work}\label{sec:conclusion}

% \subsection{Ideas about future work}

% \begin{itemize}
%   \item Algebra-coalgebra pairs as a way to interpret inductive data types and their
%         recursive control-flow in low-level categories. (this work)
%   \item Coherence conditions on an algebra-coalgebra pair to ensure that a chosen
%         representation is faithful. (this work ?)
%   \item Custom shortcuts for derived transformations based on the algebra-coalgebra
%         pairs, for fine-tuning the representation of compound operations.
%   \item Algebra-coalgebra pair generators for equivalence classes of isomorphic data
%         types, to automatically generate representations of commonly seen structures.
%   \item Solving for the representation that optimises some metric of a chosen set of
%         operations (e.g. space complexity, time complexity, constant factors etc.)
%         through various static and dynamic techniques
%   \item Restriction of the context category of the source language in terms of its
%         monoidal structure, to prevent certain low-level operations from being
%         expressible at all.
%   \item Relaxing well-foundedness, to model more complicated control-flow structures
%         such as coroutines, continuations, and so on.
% \end{itemize}
