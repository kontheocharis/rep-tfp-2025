\section{Related work}

Using inductive types as a form of abstraction was first explored by Wadler
\cite{Wadler1987-zp} through \emph{views}. The extension to dependent types
was developed by McBride and McKinna \cite{Mcbride2004-fd}, as part of the
Epigram project. Our system differs from views in the computational content
of the abstraction; even with deforestation \cite{Wadler1990-yo} views are not
always zero-cost, but representations are.
Atkey \cite{Atkey2011-ex} shows how to generically derive inductive types which
are refinements of other inductive types. This work could be integrated in our
system to automatically generate representations for refined data types.
Zero-cost data reuse when it comes to refinements of inductive types has been
explored in the context of Church encoding in Cedille \cite{Diehl2018-ba}, but
does not extend to custom representations.

Work by Allais \cite{Allais2023-pf,Allais2023-zq} uses a combination of views,
erasure by quantitative type theory, and universes of flattened data types to
achieve performance improvements when working with serialised data in Idris 2.
Our approach differs because we have access to `native' data representations, so
we do not need to rely on encodings. Additionally, they rely on heuristic
compiler optimisations to erase their views. On the topic of memory layout
optimisation, Baudon \cite{Baudon2023-cy} develops Ribbit, a DSL for the
specification of the memory representation of algebraic data types, which can
specify techniques like struct packing and bit-stealing. To our knowledge
however, this does not provide control over the indirection introduced by
\emph{inductive} types.

Dependently typed languages with extraction features, including Idris 2 \cite{idris-extraction}, Rocq
\cite{coq-extraction} and Agda \cite{agda-extraction}, have some overlapping
capabilities with our approach, but they do not provide any of the correctness
guarantees. Optimisation tricks such as the Nat-hack, and its generalisation to
other types, can emulate a part of our system but are unverified and special
casing in the compiler. Since the extended abstract version of this paper, an
optimisation was merged into Idris 2 \cite{idris-pr} to erase the forgetful and
recomputation functions for reindexing list/maybe/number-like types.
There is also demand for this kind of optimisation in Agda
\cite{agda-issue}.

% We hope
% that this work will inspire similar optimisations in other dependently typed
% languages, for which there seems to be a demand.

\section{Future work}

% We could also
% formalise the relationship between inductive families in $\lambdaind$ and the
% standard formulation in terms of $W$-types \cite{Abbott2004-va}.

There are elements of our formalisation which should be developed further. We
did not formulate normalisation and decidability of equality for $\lambdadata$,
which is needed for typechecking. We have implemented a
normalisation-by-evaluation algorithm used in \superfluid, but have only
sketched that it has the desired properties. On the practical side, we have not
explored examples of representations in great detail. Once the implementation of
$\superfluid$ is more complete, we would like to explore more sophisticated and
complete examples, along with compelling benchmarks. We would also like to
describe \superfluid's feature of representing global function definitions more
formally in the future.

As a next step we aim to expand the class of theories we consider, in particular
to include quotient-induction. Representations for quotient-inductive types in
could give rise to ergonomic ways of computing with more `traditional' data
structures such as hash maps or binary search trees. We could program
inductively over these structures but extract programs without redundancy in
data representation. Additionally, quotient-inductive types could be a good
candidate for improving typechecking ergonomics by deciding their equational
theories, similar to Frex \cite{Allais2023-rg}. Such a system could
apply certain representations during typechecking rather than code generation,
to solve equations involving free variables through a normalisation-by-evaluation  \cite{Altenkirch2020-rm}
procedure.

% We could also look into automating the discovery of inductive algebras for
% `known' classes of data types through metaprogramming \cite{Dagand2017-nj}. This
% could reproduce optimisation techniques by modern proof assistants but as
% part of a standard library, with accompanying internal correctness proofs. It
% could also extend to identity function detection, by internalising the
% extraction step explored in \cref{sub:irr}.

We would like to further explore the relationship of the $\mta{Repr}$ modality with
general systems for defining modalities, such as \emph{multi-modal type theory}
by Gratzer \cite{Gratzer2020-kf}. Additionally, we expect that metaprogramming with
representations is most naturally done in the context of \emph{two-level type theory} (2LTT)
\cite{Kovacs2022-vb}. We would like to explore the embedding of $\lambdadata$
in a two-level type theory, where signatures become types in the meta-fragment.
Then we could develop various methods of generating representations internally,
for example through algebraic ornaments, without needing to laboriously prove
induction principles by hand. The translation step in \cref{sub:translation}
can already be viewed as a kind of staging procedure, and could potentially be
integrated with the one of 2LTT.

\section{Conclusion}

This paper addresses some of the inefficiencies of inductive families in
dependently typed languages by introducing custom runtime representations that
preserve logical guarantees and simplicity of the surface language while
optimising performance and usability. These representations are formalised as inductive
algebras, and come with a framework for reasoning about them: provably zero-cost
conversions between original and represented data.
The compilation process guarantees erasure of abstraction layers, translating
high-level constructs to their defined implementations. Our hope is that by
decoupling logical structure from runtime representation, type-driven
correctness can be leveraged further without great sacrifices in performance.


\subsubsection{Acknowledgements}
We thank the anonymous reviewers for their feedback that
significantly improved the quality of this paper, Guillaume Allais for helpful
comments on the extended abstract, as well as Nathan Corbyn, Ellis
Kesterton, and Matthew Pickering for interesting discussions surrounding it.
