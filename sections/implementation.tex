\section{Implementation}\label{sec:implementation}

\superfluid is a programming language with dependent types with quantities,
inductive families and data representations. Its compiler is written in Haskell
and the compilation target is JavaScript. After prior to code generation, the
$\R$ transformation is applied to the elaborated core program, which erases all
inductive constructs with defined representations. Then, a JavaScript program is
extracted, erasing all irrelevant data by usage analysis similarly to Idris 2.
As a result, with appropriate postulates in the prelude, we are able to
represent $\datalab{Nat}$ as JavaScript's \texttt{BigInt}, and $\datalab{List}\
T$/$\datalab{SnocList}\ T$/$\datalab{Vec}\ T\ n$ as JavaScript's arrays with the
appropriate index refinement, such that we can convert between them without any
runtime overhead. The syntax of \superfluid very closely mirrors the syntax
given in the first half of this paper. It supports global definitions, inductive
families, as well as postulates. Users are able to define custom representations
for data types using \texttt{repr} blocks as defined earlier.

Currently we do not require proofs of eliminator coherence, but they are
straightforward to add. We also treat the rule $\mta{Repr-C}_i$
(\eqref{eq:repr-ci}) rule as definitional in the implementation, at the cost of
breaking confluence, but with the benefit of fewer manual transports. We are
currently working on adding dependent pattern matching that is elaborated to
internal eliminators, so that we can take advantage of the structural
unification rules for data types \cite{McBride2006-fp}.
We have written some of the examples in this paper in \superfluid, which can be
found in the \texttt{examples} directory. Overall the implementation is a proof of concept,
but we expect that our framework can be implemented in an existing language.
