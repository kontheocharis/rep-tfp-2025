\section{Introduction}

Let $\cat S$ and $\cat L$ be two 2-categories. Eventually we will require
additional structure to be present in these, but for now we want to consider
them as context categories for two different type theories. We will consider
2-morphisms to be rewriting rules, and 1-morphisms to be terms, in the style of
  [CITE].

We want to view $\cat S$ as the "high-level", nice category that we want to
write our programs in. On the other hand, $\cat L$ is the "low-level", ugly
category that we actually execute our programs in. It is ugly from the point of
view of the programmer, but nice from the point of view of a machine, because
it closely follows the way the machine actually executes the program.

Since we have a high-level and a low-level category, we want to have a way to
translate between them. We will do this by defining two functors:
\begin{itemize}
  \item A functor $\fun C : \cat S \funArrow \cat L $, called the \emph{compilation}
        functor. It takes a program in $\S$ and compiles it to a program in $\cat L$.
  \item A functor $\fun Q : \cat L \funArrow \cat S $, called the \emph{quoting}
        functor. it allows us to opaquely operate on program fragments from $\cat L$
        within $\cat S$.
\end{itemize}

Importantly, both of these functors are \emph{lax}, meaning that they only
respect functoriality up to 2-morphisms. In other words, compiling a high-level
program $\fun C(f \circ g)$ is not necessarily the same as compiling $\fun
  C(f)$ and then $\fun C(g)$ separately, but there is a 2-morphism $\fun C(f)
  \circ \fun C(g) \Rightarrow \fun C(f \circ g)$ implying that one evaluates to
the other.

The quoting functor $\fun Q$ is left-adjoint to the compilation functor $\fun
  C$, in the sense that compiling a quoted program yields the same program.
Commonly this adjunction is strict.

\section{Features of $\cat S$}

Since $\cat S$ is the site of our high-level programs, we want it to have a lot
of the things that we expect from a programming language. In particular, we
want it to be cartesian closed, and we want it to admit fixpoints of
endofunctors. Furthermore, we expect that there exists a terminal object $1 \in
  \Ob(\cat S)$.

\section{Features of $\cat L$}

We require much less convenience in the structure of $\cat L$, instead relying
on the functor $\fun C$ to translate human-friendly programs into
machine-friendly ones.

Commonly $\cat L$ will not be closed, and it might be a Kleisli category with
respect to some monad (probably a monad holding the state of the
stack/heap/environment/etc).

We will commonly want to restrict $\cat L$ to only be monoidal in some
less-than-cartesian way, for example to model a linear logic for the tracking
and preservation of resources.

\section{Compilation}

This setup of having two sites of "programming" is not a new concept. In fact,
it could be argued that this is exactly how the process of standard compilation
works. We have a high-level language, which is compiled into a low-level
language, which is then executed on a machine.

However, the process of compilation is usually not considered to be part of the
high-level program itself. Rather, it is the job of the compiler to figure out
exactly what is the optimal representation of a high-level program in the
low-level language. This is especially the case when the discrepancy between
the levels becomes great, for example in functional languages such as Haskell
or ML. In these languages, the programmer has very little control over the
eventual representation of their program in the low-level language, unless they
use specialised primitives which are meant to mirror the low-level capabilities
of the target system.

This is what we are aiming to change with this formalism. We envision a process
of compilation in which the programmer has a lot more control over the
representation of their program in the low-level language, while still keeping
a clear separation between that and the core logic of the high-level program
itself. In a way, we are trying to create a principled and ubiquitous version
of "directive annotations" in high-level languages which aim to address a
particular low-level concern about a program (for example, whether to keep the
field of a data type behind a pointer.)

\section{Translation of logic and data}

In the present categorical presentation of this process, we want each program
to construct its own customised compilation functor $\fun C$. This functor will
be constructed by the programmer, just like the rest of the high-level logic of
the program.

One might remark that this sounds like a lot of work for the programmer.
Imagine having to not only write a program, but also write a customised
compiler for that very program, and doing that each time you want to write a
program. However, we will see that this is not the case. In fact, we will see
that the compilation functor can be constructed automatically based on a set of
core "data type representations" that are provided by the programmer. These
data type representations will be the only thing that the programmer will have
to write, and they will be reusable across many different programs.

Furthermore, there will be a trade-off at play. The more custom parts of the
compilation functor the programmer writes, the more control they have over the
eventual representation, but the more coupling there is between the high-level
program and the low-level representation. Crucially however, this coupling is
never present in the high-level program itself, but rather sits alongside it.

\section{Algebra-coalgebra pairs as the primitive of representation}

Consider that we have an endofunctor $\fun F : \cat S \funArrow \cat S$ which
admits a least fixpoint, which we will denote $\mu \fun F$. This is a certain
kind of free structure in $\cat S$: an inductive data type. We will mostly
focus on least fixpoints, however it might be worth considering the dual case
of greatest fixpoints as well.

We want to be able to represent this in $\cat L$. We will do this by defining
an algebra and a coalgebra in $\cat S$ over a chosen object $I_{\fun F} \in
  \Ob(\cat S)$,
\begin{equation}
  \begin{tikzcd}
    F(I_{\fun F}) \arrow[r, "w", shift left] & I_{\fun F} \arrow[l, "u", shift left]
  \end{tikzcd}.
\end{equation}
This pair of morphisms allows us to "peel away" a layer of $\fun F$ from $I_{\fun F}$, as
well as "wrap" a layer of $\fun F$ around $I_{\fun F}$. Conceptually, $I_{\fun F}$ represents an
intermediary descriptive object which is used to gather and encode data about
the structure of an inhabitant of $\mu F$.

Furthermore, $I_{\fun F}$ must satisfy a second property: there must exist
another pair of morphisms
\begin{equation}
  \begin{tikzcd}
    I_{\fun F} \arrow[r, "r", shift left] & \fun Q(R_{\fun F}) \arrow[l, "i", shift left]
  \end{tikzcd}.
\end{equation}
with the quoted object $\fun Q(R_{\fun F})$, making it so that $R_{\fun F} \in \Ob(\cat L)$ is the chosen low-level
representation of $\mu \fun F$.

In general we do not expect that $r$ and $i$, or $w$ and $u$ are strict
inverses. In fact, them being strict inverses would imply a very trivial
correspondence that probably does not contain significant information. However,
we do expect that there exist two pairs of 2-morphisms
\begin{equation}
  \nu : u \circ w \Rightarrow \mathrm{id}_{\fun F(I_{\fun F})} \text{ and } \nu' : w \circ u \Rightarrow \mathrm{id}_{I_{\fun F}}
\end{equation}
and
\begin{equation}
  \mu : i \circ r \Rightarrow \mathrm{id}_{I_{\fun F}} \text{ and } \mu' : r \circ i \Rightarrow \mathrm{id}_{\fun Q(R_{\fun F})} \,.
\end{equation}
TODO: do we really need both directions here?
These 2-morphisms are the key to the correspondence between the high-level and
low-level representations. They ensure that no matter how different the composition of the
encoding and decoding morphisms is from "doing nothing", the result always evaluates to the same term.

% For example, consider $F(X) = 1 + AX$ for some $A \in \Ob(\S)$. Then we can
% define $I_F = $, with

\section{An example: Lists as heap arrays}

We will define and work in the internal languages of $\cat S$ and $\cat L$.

For $\cat S$:

\begin{align}
  x,y  \tag{variables}                                                                                                                                                                 \\
  A, B    & ::=  A \to B \mid A \times B \mid A + B \mid \mu x . A \mid 1 \tag{types}                                                                                                  \\
  t, u, v & ::= \lambda x . t \mid x \mid t\,u \mid (t, u) \mid \int{p}_1(t) \mid \int{p}_2(t) \mid \int{inl}(t) \mid \int{inr}(t) \mid \int{case}(t, x.u, y.v) \mid \star \tag{terms}
\end{align}

For $\cat L$:

\begin{align}
  x  \tag{variables}                                                                                                                                                                               \\
  n \tag{finite natural numbers, word size}                                                                                                                                                        \\
  s    & ::= \varnothing \mid [s, t] \tag{sequences}                                                                                                                                               \\
  A, B & ::= A \multimap B \mid \Sigma x . A \mid \int{W} \mid A ^n \mid \square A \mid \int{I} \tag{types}                                                                                        \\
  t, u & ::= \lambda x . t \mid x \mid t\,u \mid \langle n, t \rangle \mid n \mid s \mid \int{box}(t) \mid \int{letbox}(t, x.t) \mid \int{letseq}(t, x.t) \mid \int{num}(t) \mid \star \tag{terms}
\end{align}

\section{Slice categories and dependent types}
