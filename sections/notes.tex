\section{[TEMP] Notes}\label{sec:notes}

\section{Document plan}

We probably need to have the following sections:

\begin{itemize}
  \item Introduction
  \item Preliminaries -- category theory, conventions, basic structure of the problem
  \item Categorical model of (compilation?) Maybe it should be called something else
  \item An example with the simply-typed lambda calculus and a language with explicit
        boxing
  \item Embedding the representations into the high-level language -- dependent types
  \item Related work
  \item Conclusion + future
  \item Would be nice to have an artifact: Agda formalisation?
\end{itemize}

% Consider that we have an endofunctor $\fun F : \cat S \funArrow \cat S$ which
% admits a least fixpoint, which we will denote $\mu \fun F$. This is a certain
% kind of free structure in $\cat S$: an inductive data type. We will mostly
% focus on least fixpoints, however it might be worth considering the dual case
% of greatest fixpoints as well.

% We want to be able to represent this in $\cat L$. We will do this by defining
% an algebra and a coalgebra in $\cat S$ over a chosen object $I_{\fun F} \in
%   \Ob(\cat S)$,
% \begin{equation}
%   \begin{tikzcd}
%     F(I_{\fun F}) \arrow[r, "w", shift left] & I_{\fun F} \arrow[l, "u", shift left]
%   \end{tikzcd}.
% \end{equation}
% This pair of morphisms allows us to "peel away" a layer of $\fun F$ from $I_{\fun F}$, as
% well as "wrap" a layer of $\fun F$ around $I_{\fun F}$. Conceptually, $I_{\fun F}$ represents an
% intermediary descriptive object which is used to gather and encode data about
% the structure of an inhabitant of $\mu F$.

% Furthermore, $I_{\fun F}$ must satisfy a second property: there must exist
% another pair of morphisms
% \begin{equation}
%   \begin{tikzcd}
%     I_{\fun F} \arrow[r, "r", shift left] & \fun Q(R_{\fun F}) \arrow[l, "i", shift left]
%   \end{tikzcd}.
% \end{equation}
% with the quoted object $\fun Q(R_{\fun F})$, making it so that $R_{\fun F} \in \Ob(\cat L)$ is the chosen low-level
% representation of $\mu \fun F$.

% In general we do not expect that $r$ and $i$, or $w$ and $u$ are strict
% inverses. In fact, them being strict inverses would imply a very trivial
% correspondence that probably does not contain significant information. However,
% we do expect that there exist two pairs of 2-morphisms
% \begin{equation}
%   \nu : u \circ w \Rightarrow \mathrm{id}_{\fun F(I_{\fun F})} \text{ and } \nu' : w \circ u \Rightarrow \mathrm{id}_{I_{\fun F}}
% \end{equation}
% and
% \begin{equation}
%   \mu : i \circ r \Rightarrow \mathrm{id}_{I_{\fun F}} \text{ and } \mu' : r \circ i \Rightarrow \mathrm{id}_{\fun Q(R_{\fun F})} \,.
% \end{equation}
% TODO: do we really need both directions here?
% These 2-morphisms are the key to the correspondence between the high-level and
% low-level representations. They ensure that no matter how different the composition of the
% encoding and decoding morphisms is from "doing nothing", the result always evaluates to the same term.

% For example, consider $F(X) = 1 + AX$ for some $A \in \Ob(\S)$. Then we can
% define $I_F = $, with

% The category $\Repr ^{\cat S} _{\cat L}$ is defined as follows:
% \begin{itemize}
%   \item Objects are 7-tuples $(F, R, I, w, u, r, i)$ (TODO: write as a diagram).
%   \item Morphisms are morphisms in $\cat S \times \cat L$ of the form $(F, R) \to (F',
%           R')$.
% \end{itemize}

% Given some category $M$, we consider a functor $F : M \to [\cat S, \cat S]$
% representing a $M$-indexed family of least-fixpoint objects in $\cat S$, of the
% form $\mu X . F_m$ for each $m \in M$ (and what are the morphisms looking
% like?).

% For each of the $F_m$, we expect a corresponding object in the category $\Repr
%   ^{\cat S} _{\cat L}$, denoted by $\hat{F}_m$. In other words, we expect a
% functor $\hat{F} : M \funArrow \Repr ^{\cat S} _{\cat L}$ such that $F(m)$ is
% the first element of the 7-tuple $\hat{F}(m)$.

% Given that we have some $\mu X . A \in \cat S$ such that there exists a $a \in
%   M$ for which $\pi_F (\hat{F}(a)) = A$, then we define $C(\mu X . A) = \pi_R
%   (\hat{F}(a))$.
